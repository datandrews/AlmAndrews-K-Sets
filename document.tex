\documentclass{article}

\usepackage[utf8]{inputenc}
\usepackage[]{amsmath,amsthm,amssymb,amsfonts,graphicx}
%\usepackage[backend=bibtex]{biblatex}
%\usepackage{beton}
%\usepackage{euler}
\usepackage{fourier}

\usepackage[colorinlistoftodos]{todonotes}

\theoremstyle{definition}
\newtheorem{theorem}{Theorem}%[section]
\newtheorem{definition}[theorem]{Definition}
\newtheorem{lemma}[theorem]{Lemma}
\newtheorem{corollary}[theorem]{Corollary}
\newtheorem{conj}{Conjecture}

\def\M{\mathcal{M}}

\begin{document}

We consider the complete graph $G$ on the size $7$ subsets of the set of
\emph{points} $\{1,2,\dots,14\}$. That is, $V = V(G) = \binom{[14]}{7}$. We
provide an edge-coloring that is good with respect to $\M_{2}$ based the size
of intersection of the sets at each vertex. Suppose $X$ and $Y$ are distinct
elements of $\binom{[14]}{7}$.  Define $c$ by
\begin{itemize}
\item if $|X \cap Y| = 0$ then $c(XY) = b_{0}$;
\item if $|X \cap Y| = 1$ then $c(XY) = b_{1}$;
\item if $|X \cap Y| = 2$ then $c(XY) = b_{i}$ for some $i = 0,1$ as described
  below;
\item if $|X \cap Y| \geq 3$ then $c(XY) = r$.
\end{itemize}
A useful notion is that of a \emph{$j$-edge}: if $|X \cap Y| = j$ then we call
$XY$ a $j$-edge. Then the above coloring can be summarized as: all 0-edges are
$b_{0}$, all 1-edges are $b_{1}$, all 3-and-higher-edges are $r$, and all
2-edges are either $b_{0}$ or $b_{1}$ based on a rule to be determined.

Immediately we can show that this coloring has no blue triangles: if $X_{1}$,
$X_{2}$, and $X_{3}$ form an all-blue triangle, then for $i \neq j$, $|X_{i}
\cap X_{j}| \leq 2$ so
\begin{equation*}
  |X_{1} \cup X_{2} \cup X_{3}|
  = \sum_{i} |X_{i}|  - \sum_{i\neq j} |X_{i} \cap X_{j}| + |X_{1} \cap X_{2} \cap X_{3}|  \geq 15
\end{equation*}
a contradiction since there are only $14$ points.

To determine color of the 2-edges we will use a \emph{splitting graph},
$G_{s}$. This is a complete graph on the points $[14]$ with all edges colored
either $b_{0}$ or $b_{1}$. We use this coloring to induce a coloring on the
2-edges of $E$: if $X \cap Y = \{i, j\}$ with $i,j$ distinct (so that $XY$ is a
2-edge) then $c(XY)$ will be the color of the edge $ij$ in $G_{S}$.

% \todo[color=green,inline]{The following paragraph needs work!}
% It is important to note that while the coloring of $G_{s}$ is used to color
% some of the edges in $G$, it is a different coloring entirely.  In fact, we
% will see in the following that the edge-coloring of $G_{s}$ needs to satisfy
% three subgraph color properties: no $K_{4}$ subgraph, $K_{3,3}$ subgraph, nor
% $K_{5,2}$ subgraph of $G_{s}$ can be monochromatic.

It is important to note that while the coloring of $G_{s}$ is used to color
some of the edges in $G$, specifically the 2-edges, the graphs $G$ and $G_{s}$
are entirely different graphs. In the argument below we will need a coloring of
the edges of $G_{s}$ that contains no monochromatic $K_{4}$, no monochromatic
$K_{4,3}$, and no monochromatic $K_{5,2}$. Graphs with all three of these
properties do exist. For example, we consider consider the cyclic coloring of
$K_{17}$ in which one of the of the color classes (which we will color $b_{0}$)
is given by $\mathcal{C}_{0} = \{1,2,4,8,9,13,15,16\}$ and the other (which we
will color $b_{1}$) is given by $\mathcal{C}_{1} = \{3,5,6,7,10,11,12,14\}$.
It can be verified that this coloring has no monochromatic $K_{4}$, $K_{4,3}$,
or $K_{5,2}$ subgraphs, and so for $G_{s}$ we could use the subgraph obtained
from this $K_{17}$ with vertices 15, 16, and 17 deleted.

For example, let
\begin{equation*}
  X =\{1, 2, 3, 4, 5, 6, 7\},\\
  Y =\{2, 3, 4, 5, 6, 7, 8\},\\
  Z =\{7, 8, 9, 10, 11, 12, 13\}
\end{equation*}
Consider the triangle $XYZ $.  Since $|X\cap Y| = 6 \geq 3$, $c(XY)=r$. Since
$|X\cap Z| = 1$, $c(XZ)=b_1$. Since $|y\cap Z| = 2$, we must look at $G_s$. We
have $Y\cap Z=\{7,8\}$, so we look at the color of the edge between
the vertices labeled 7 and 8 in the example $G_{s}$ given above. Since $8-7=1\in\mathcal{C}_0$, $c(YZ)=b_0$.

\todo[color=green,inline]{Notation from choices:
  $\{1,2,3,4,5,[8,9,10,11]_{2}\}$,
  $\{[1,2,3,4]_{2},[8,9,10,11]_{2},12,13,14\}$,
  $\{[1,2]_{1},[3,4,5,6,7]_{1},10,11,12,13,14\}$}

To show that each edge in $G$ has all its needs met, we will consider without
loss of generality only edges between the vertex $X = \{1, 2, 3, 4, 5, 6, 7\}$
and a vertex $Y$ that overlaps $X$ in its last $j$ elements. For example, for
$j=3$ we will consider $X = \{1,2,3,4,5,6,7\}$ and $Y = \{5,6,7,8,9,10,11\}$.
Furthermore, in order to not loose any generality in the proof, we will not
assume anything about the particular coloring of $G_{s}$ other than the
monochromatic-subgraph-free properties given above.

For $j=0,1,$ and $2$, the edge $XY$ is colored either $b_{0}$ or $b_{1}$, and
has five needs. If $c(XY)=b_{i}$ these needs are $(b_{i},r,b_{0})$,
$(b_{i},r,b_{1})$, $(b_{i},b_{0},r)$ and $(b_{i},b_{1},r)$, and $(b_{i},r,r)$.
For $j=3,4,5,$ and $6$, the edge $XY$ is colored $r$ and has 9 needs, of four
different categories:
\begin{itemize}
\item the \emph{homogeneous blue} needs: $(r,b_{0},b_{0})$ and $(r, b_{1}, b_{1})$;
\item the \emph{heterogeneous blue} needs: $(r,b_{0},b_{1})$ and $(r,b_{1},b_{0})$;
\item the \emph{red-blue} needs: $(r,r,b_{i})$ and $(r,b_{i},r)$ for $i \in
  \{0,1\}$; and
\item the \emph{all red} need: $(r,r,r)$.
\end{itemize}

\subsection{$0$-edge case}

We work out the details carefully in this case as the other cases use similar
approaches.

For this case, we are considering $X = \{1,2,3,4,5,6,7\}$ and $Y =
\{8,9,10,11,12,13,14\}$. By our coloring $c(XY) =b_{0}$ and so we must show
there are vertices that witness the five needs: $(b_{0},r,b_{0})$,
$(b_{0},r,b_{1})$, $(b_{0},b_{0},r)$ and $(b_{0},b_{1},r)$, and $(b_{0},r,r)$.

To show that the first need is satisfied,
% it suffices to find a vertex $Z_{1} \in V$ different from $X$ and $Y$ such
% that $c(XZ_{1}) = r$ and $c(YZ_{1}) = b_{0}$.  The first requirement is easy
% to satisfy, as we must simply make sure that $|X \cap Z_{1}| \geq 3$. There
% are two ways to get $c(YZ_{1}) = b_{0}$: we could make $Y$ and $Z_{1}$
% disjoint, or we could make $Y \cap Z_{1} = \{i,j\}$ such that color of the
% $ij$ edge in $G_{s}$ is $b_{0}$.  In this case it is impossible to make $Y$
% and $Z_{1}$ disjoint, since $X$ is the only such set, so we must find an
% appropriate $i,j$ combination to put into $Z_{1}$.
we consider selecting $Z$ from the 6-set collection
$\{1,2,3,4,5,[8,9,10,11]_{2}\}$. All choices from this collection overlap $X$
in 5 points --- giving $c(XZ) = r$ --- and overlap $Y$ in 2 points so that the
edge $YZ$ is colored the same as the corresponding edge in $G_{s}$. The
subgraph of $G_{s}$ induced by the points $\{8,9,10,11\}$ form a $K_{4}$
subgraph, and so is not monochromatic. Thus there must be an edge $ij$ in
this subgraph that is colored $b_{0}$. Thus we can use $Z=\{1,2,3,4,5,i,j\}$ to
witness this first need. Similarly, there must be an edge $k\ell$ in this subgraph
that has color $b_{1}$, and so $Z' = \{1,2,3,4,5,k,\ell\}$ that satisfies the
second need: $c(XZ') = r$ and $c(YZ') = b_{1}$.

A similar construction can be used to find vertices that witness the
$(b_{0},b_{i},r)$ needs.  The need $(b_{0},r,r)$ is also witnessed, and so we
summarize:
\begin{itemize}
\item $(b_{0},r,b_{0})$ and $(b_{0},r,b_{1})$ have witnesses from
  $\{1,2,3,4,5,[8,9,10,11]_{2}\}$; 
\item $(b_{0},b_{0},r)$ and $(b_{0},b_{1},r)$ have witnesses from
  $\{[1,2,3,4]_{2},8,9,10,11,12\}$; and
\item $(b_{0},r,r)$ is satisfied by the witness $\{5,6,7,8,9,10,14\}$.
\end{itemize}
Thus all needs of all $0$-edges are met in our graph coloring.

\subsection{$1$-edge case}

We consider $X = \{1,2,3,4,5,6,7\}$ and $Y =
\{7,8,9,10,11,12,13\}$, so that $c(XY) = b_{1}$. In this case, we can provide
witnesses for all 5 needs by direct construction:
\begin{itemize}
\item $(b_{1},r,b_{0})$ is witnessed by $\{1,2,3,4,5,6,14\}$;
\item $(b_{1},r,b_{1})$ is witnessed by $\{1,2,3,4,5,6,8\}$;
\item $(b_{1},b_{0},r)$ is witnessed by $\{8,9,10,11,12,13,14\}$;
\item $(b_{1},b_{1},r)$ is witnessed by $\{6,8,9,10,11,12,13\}$; and
\item $(b_{1},r,r)$ is again witnessed by $\{5,6,7,8,9,10,14\}$.
\end{itemize}

\subsection{$2$-edge case}

For this case, we consider $X = \{1,2,3,4,5,6,7\}$ and $Y =
\{6,7,8,9,10,11,12\}$. We know $c(XY) = b_{i}$ for either $i=0$ or $1$, but
which one is not relevant, as the form of the needs to do not differ, and can
be satisfied again by direct construction:
\begin{itemize}
\item $(b_{i},r,b_{0})$ is witnessed by $\{1,2,3,4,5,13,14\}$;
\item $(b_{i},r,b_{1})$ is witnessed by $\{1,2,3,4,5,8,14\}$;
\item $(b_{i},b_{0},r)$ is witnessed by $\{8,9,10,11,12,13,14\}$;
\item $(b_{i},b_{1},r)$ is witnessed by $\{7,8,9,10,11,12,13\}$; and
\item $(b_{i},r,r)$ is also witnessed by $\{5,6,7,8,9,10,14\}$.
\end{itemize}

\subsection{$3$-edge case}

Since $X = \{1,2,3,4,5,6,7\}$ and $Y = \{5,6,7,8,9,10,11\}$ we have $c(XY) =
r$.  The red-blue needs are satisfied in a similar way as they were in the
$0$-edge case: the $(r,r,b_{i})$ needs are witnessed by some vertices selected
from $\{1,2,3,[8,9,10,11]_{2},13,14\}$, and the $(r,b_{i},r)$ are witnessed by
vertices in $\{[1,2,3,4]_{2},9,10,11,13,14\}$.

The homogeneous and heterogeneous blue needs all have witnesses in
$\{[1,2,3,4]_{2}, [8,9,10,11]_{2}, 12, 13, 14\}$: the subgraph of $G_{s}$
induced by $\{1,2,3,4\}$ has $b_{0}$ and $b_{1}$ edges, as does the subgraph
induced by $\{8,9,10,11\}$. Furthermore, edges from both of these subgraphs can
be selected independently since the first only involves points from $X$ and the
second only points from $Y$. Thus, we can find witnesses for all of these
needs.

In summary, we have:
\begin{itemize}
\item $(r,b_{i},b_{j})$ for $i,j \in \{0,1\}$) have witnesses in $\{[1,2,3,4]_{2}, [8,9,10,11]_{2}, 12, 13, 14\}$;
\item $(r,r,b_{i})$ have witnesses in $\{1,2,3,[8,9,10,11]_{2},12,13\}$;
\item $(r,b_{k},r)$ have witnesses in $\{[1,2,3,4]_{2},8,9,10,12,13\}$; and
\item $(r,r,r)$ has witness $\{5,6,7,8,9,10,14\}$.
\end{itemize}

\subsection{$4$-edge case}


For this case, $X = \{1,2,3,4,5,6,7\}$ and $Y = \{4,5,6,7,8,9,10\}$, and $c(XY)
= r$.

We begin with the easiest needs, red-blue and all red:
\begin{itemize}
\item The need $(r,r,b_{0})$ is witnessed by $\{1,2,3,11,12,13,14\}$;
\item The need $(r,r,b_{1})$ is witnessed by $\{1,2,3,10,11,12,13\}$;
\item The need $(r,b_{0},r)$ is witnessed by $\{8,9,10,11,12,13,14\}$;
\item The need $(r,b_{1},r)$ is witnessed by $\{1,9,10,11,12,13,14\}$; and
\item The need $(r,r,r)$ is witnessed by $\{5,6,7,8,9,10,14\}$.
\end{itemize}

To deal with the homogeneous and heterogeneous blue needs, we will look at the
subgraphs of $G_{s}$ induced by by $\{1,2,3\}$ and $\{8,9,10\}$: call these
$T_{X}$ and $T_{Y}$ respectively.  We proceed by cases:
\begin{description}
\item[Case I:] neither $T_{X}$ nor $T_{Y}$ is monochromatic. Then each blue-only
  need is witnessed by a vertex in $\{[1,2,3]_{2},[8,9,10]_{2},11,12,13\}$.
\item[Case II:] $T_{X}$ is monochromatic $b_{0}$.  

  % We will use the following two facts in the subcases below:

  % For any $i \in \{4,5,6,7\}$, the subgraph induced by $\{1,2,3,i\}$ not
  % monochromatic. But this means, since $T_{x}$ is monochromatic, there must be
  % an $\ell \in \{1,2,3\}$ such that the edge $i\ell$ is $b_{1}$.  Thus each of
  % the points in $\{4,5,6,7\}$ must have a $b_{1}$ edge to some point in
  % $T_{X}$.

  % Consider the bipartite subgraph of $G_{s}$ induced by $L=\{1,2,3\}$ and
  % $R=\{4,5,6,7\}$. Since this is not monochromatic, so there must be one edge
  % $ij$ ($i \in L$, $j \in R$) that is $b_{0}$ and one edge $mn$ ($m \in L$, $n
  % in R$) that is $b_{1}$.

  \begin{description}
  \item[Case II.A:] $T_{Y}$ is not monochromatic. In this case there are
    vertices in $\{1,2,[8,9,10]_{2},12,13,14\}$ that witness to the needs
    $(r,b_{0},b_{i})$ for $i=0,1$. 

    We consider the $K_{4,3}$ subgraph of $G_{s}$ induced by $L=\{4,5,6,7\}$
    and $R=\{8,9,10\}$. Since this subgraph is not monochromatic, there must be
    an edge $ij$ ($i \in L$, $j \in R$) that is $b_{0}$. Now consider the
    subgraph of $G_{s}$ induced by $\{1,2,3,i\}$, which is not monochromatic.
    Since $T_{X}=\{1,2,3\}$ is monochromatic $b_{0}$, there must be an edge
    $ik$ ($k \in \{1,2,3\}$) that is $b_{0}$.  We now have a witness to the
    $(r,b_{0},b_{0})$ need: $\{k,i,j,11,12,13,14\}$.

    In a similar way, there must be an edge $mn$ ($m \in L$, $n \in R$) that is
    $b_{1}$; considering the subgraph induced by $\{1,2,3,m\}$ there must be an
    $\ell \in \{1,2,3\}$ such that edge $\ell{}m$ is $b_{0}$, and so $\{\ell,
    m, n, 11, 12, 13, 14\}$ is a witness for $(r,b_{0},b_{1})$.
  \item[Case II.B:] $T_{Y}$ is monochromatic $b_{0}$. We immediately have a
    witness for $(r, b_{0}, b_{0})$: $\{1,2,9,10,11,12,13\}$.

    As in the previous case, we consider the bipartite subgraph induced by
    $L=\{4,5,6,7\}$ and $R=\{8,9,10\}$, which must have an edge $ij$ ($i \in
    L$, $j \in R$) that is $b_{1}$.  Furthermore, because the subgraph induced
    by $\{1,2,3,i\}$ is not monochromatic (but $T_{X}$ is monochromatic
    $b_{0}$), there must be an edge $i\ell$ ($\ell \in \{1,2,3\}$) that is
    $b_{1}$.  Thus $\{\ell, i, j, 11, 12, 13, 14\}$ is a witness for $(r,
    b_{1}, b_{1})$.

    If we again consider the subgraph induced by $L=\{4,5,6,7\}$ and
    $R=\{8,9,10\}$, there must be an edge $ij$ ($i \in L$, $j \in R$) that is
    $b_{0}$. Furthermore, $\{1,2,3,i\}$ must have an edge $i\ell$ ($\ell \in
    \{1,2,3\}$) that is $b_{1}$. Thus $\{\ell, i, j, 11, 12, 13, 14\}$ is a
    witness for $(r, b_{1}, b_{0})$.

    Finally, we consider another bipartite subgraph, that one induced by
    $L=\{4,5,6,7\}$ and $R=\{1,2,3\}$. There must be an edge $ij$ ($i in L$, $j
    \in R$) that is $b_{0}$. Since the subgraph induced by $\{8,9,10,i\}$ is
    not monochromatic (but $T_{Y}$ is monochromatic $b_{0}$) there must be an
    edge $i\ell$ ($\ell \in \{8,9,10\}$) that is $b_{1}$. Thus $\{i, j, \ell,
    11, 12, 13, 14\}$ is a witness for $(r, b_{0}, b_{1})$.
  \item[Case II.C:] $T_{Y}$ is monochromatic $b_{1}$. In this case, we
    immediately have $\{1,2,9,10,11,12,13,14\}$ as a witness for
    $(r,b_{0},b_{1})$.

    Now consider the subgraph induced by $\{1,2,3,4\}$: since this is not
    monochromatic (but $T_{X}$ is monochromatic $b_{0}$) there is an $i \in
    \{1,2,3\}$ such that edge $4i$ is $b_{1}$.  Similarly, the subgraph induced
    by $\{4,8,9,10\}$ must have an edge $4j$ that is $b_{0}$.  Thus, $\{i, j,
    4, 11, 12, 13, 14\}$ is a witness of $(r,b_{1},b_{0})$.

    Once again, we consider the bipartite subgraph induced by $L=\{4,5,6,7\}$
    and $R=\{8,9,10\}$.  There must be an edge $ij$ ($i \in L$, $j \in R$) that
    is $b_{1}$. But considering the subgraph induced by $\{1,2,3,i\}$, there
    must be an edge $i\ell$ ($\ell \in \{1,2,3\}$) that is $b_{1}$. Thus
    $\{\ell, i, j, 11, 12, 13, 14\}$ is a witness of $(r, b_{1}, b_{1})$.

    Finally, we consider the bipartite subgraph induced by $L=\{4,5,6,7\}$ and
    $R=\{1,2,3\}$. There must be an edge $ij$ ($i \in L$, $j \in R$) that is
    $b_{0}$. However, in the subgraph induced by $\{8,9,10,i\}$ there must be
    an edge $i\ell$ ($\ell in \{8,9,10\}$) that is $b_{0}$. And so, $\{\ell, i,
    j, 11, 12, 13, 14\}$ is a witness to $(r, b_{0}, b_{0})$.
  \end{description}
\end{description}

\subsection{$5$-edge case}

For this case, $X = \{1,2,3,4,5,6,7\}$ and $Y = \{3,4,5,6,7,8,9\}$, and $c(XY)
= r$.

It is straight forward to see that the red-blue, homogeneous blue, and all-red needs are
satisfied:
\begin{itemize}
\item $(r,r,b_{i})$ have witnesses in $\{1,2,[3,4,5,6]_{2},12,13,14\}$;
\item $(r,b_{k},r)$ have witnesses in $\{[3,4,5,6]_{2},8,9,12,13,14\}$; and
\item $(r,b_{0},b_{0})$ and $(r,b_{1},b_{1})$ are both satisfied by witnesses
  in $\{[3,4,5,6]_{2},10,11,12,13,14\}$.
\item $(r,r,r)$ is witnessed by $\{5,6,7,8,9,10,14\}$.
\end{itemize}

How the heterogeneous blue needs --- $(r,b_{0},b_{1})$ and $(r,b_{1},b_{0})$ ---
are satisfied depends on the coloring of the 1-2 and 8--9 edges in $G_{s}$.
If both edges are the same color, we can construct witnesses for each of the
needs, as follows:
\begin{itemize}
\item if both 1--2 and 8--9 are $b_{0}$, then $\{1,2,8,11,12,13,14\}$
  witnesses $(r,b_{0},b_{1})$ and $\{1,8,9,11,12,13,14\}$ witnesses $(r,b_{1},b_{0})$;
\item if both 1--2 and 8--9 are $b_{1}$, then $\{1,8,9,11,12,13,14\}$
  witnesses $(r,b_{0},b_{1})$ and $\{1,2,10,11,12,13,14\}$ witnesses $(r,b_{1},b_{0})$;.
\end{itemize}

If 1--2 is $b_{0}$ and 8--9 is $b_{1}$ then $\{1,2,8,9,12,13,14\}$ is a
witness of $(r,b_{0},b_{1})$. Now consider the vertices $Z$ in
$\{[3,4,5,6,7]_{1}, [8,9]_{1}, 10, 11, 12, 13, 14\}$; for all of these $c(XZ) = b_{1}$, but
$c(YZ)$ depends on the points selected. However, the bipartite subgraph
induced by the sets $L=\{3,4,5,6,7\}$ and $R=\{8,9\}$ is $K_{5,2}$ and so has
one edge $ij$ that is colored $b_{0}$, and so $\{i,j,10,11,12,13,14\}$
witnesses $(r,b_{1},b_{0})$. A very similar argument works if 1--2 is $b_{1}$
and 8--9 is $b_{0}$.

\subsection{$6$-edge case}

Our final case considers $X = \{1,2,3,4,5,6,7\}$ and $Y = \{2,3,4,5,6,7,8\}$.
The heterogeneous blue and all-red needs are satisfied by directly constructed
witnesses:
\begin{itemize}
\item $\{1,9,10,11,12,13,14\}$ is a witness of $(r,b_{1},b_{0})$;
\item $\{8,9,10,11,12,13,14\}$ is a witness of $(r,b_{0},b_{1})$;
\item $\{5,6,7,8,9,10,14\}$ is a witness of $(r,r,r)$.
\end{itemize}
The homogeneous blue needs --- $(r,b_{0},b_{0})$ and $(r,b_{1},b_{1})$ --- must
each have a witness in $\{[2,3,4,5]_{2},10,11,12,13,14\}$.  Finally the needs
$(r,r,b_{i})$ are witnessed by some vertices in
$\{1,[2,3,4,5]_{2},11,12,13,14\}$ and the needs $(r,b_{i},r)$ are witnessed by
vertices in $\{[2,3,4,5]_{2},8,11,12,13,14\}$.




\end{document}

%%% Local Variables: 
%%% mode: latex
%%% TeX-master: t
%%% End: 
